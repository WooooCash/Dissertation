% LTeX: language=pl-PL


\documentclass[a4paper, 12pt]{article} \linespread{1.3}

\usepackage[T1]{fontenc} \usepackage[utf8]{inputenc} \usepackage{indentfirst}

\usepackage[polish]{babel}

\usepackage[margin=3cm]{geometry}

\title{Animacja szkieletowa z wykorzystaniem kinematyki odwrotnej w silniku
Unity} 
\author{Łukasz Białczak} \date{}
	
\begin{document}

\maketitle

\tableofcontents

\section{Wstęp} *Wskazówki dla mnie* \begin{itemize}

    \item Omówienie podstaw animacji szkieletowej w grach komputerowych 
    \item Omówienie wybranych metod kinematyki odwrotnej 
    \item Zapoznanie się z silnikiem Unity, w szczególności komponentami
        odpowiedzialnymi za implementację animacji szkieletowej oraz kinematykę
        odwrotną 
    \item Implementacja aplikacji demonstracyjnej w silniku Unity,
        wykorzystującej kinematykę odwrotną 
    \item Porównanie animacji opartej o kinematykę odwrotną z animacją
        realizowaną w trybie wypalanym

\end{itemize}


\textbf{Opis tematu}

W klasycznej animacji szkieletowej kości reprezentowane są za pomocą pewnej
struktury drzewiastej obiektów. Wykonywanie sekwencji animacji polega zwykle na
aktualizacji parametrów tych obiektów (położenia i orientacji kości) w porządku
od korzenia do liści. W przypadku kinematyki odwrotnej, jak sama nazwa wskazuje,
porządek aktualizacji jest odwrotny tj. od danego liścia do określonego węzła
nadrzędnego. Metody te pozwalają na proceduralne tworzenie sekwencji animacji
umożliwiających wierniejsze odzwierciedlenie interakcji animowanego obiektu z
otoczeniem bez konieczności ręcznego "wypalania" sekwencji animacji (ang. baked
animation). Na przykład: podnoszenie różnych przedmiotów, obsługa urządzeń typu
dźwignie, koła, dopasowanie ustawienia kończyn postaci do nierówności terenu
itp. Znalezienie, właściwych wartości parametrów kości dla sekwencji animacji
generowanych na podstawie kinematyki odwrotnej nie zawsze jest zadaniem prostym i
czasem wymaga użycia zaawansowanych metod optymalizacyjnych. Celem pracy jest
zapoznanie się z podstawowymi algorytmami wykorzystywanymi w kinematyce
odwrotnej, jak również funkcjonalnościami oferowanymi przez silnik Unity w
zakresie ww. technik.

\section{Techniki implementacji kinematyki odwrotnej} Istnieje wiele algorytmów
służących do implementacji kinematyki odwrotnej różniące się zastosowaniem i
złożonością. Można wyróżnić prostsze implementacje, działające tylko na
hierarchiach kości składających się z kilku obiektów. Istnieją też algorytmy,
które pozwalają na wykorzystanie dowolnej długości łańcuchów, ale są bardziej
złożone matematycznie. Często jest wtedy stosowane podejście heurystyczne,
obliczające jedynie aproksymacje właściwego umiejscowienia kości w przestrzeni
trójwymiarowej. 

\[
	\sum_{n=1}^{\infty}2^n
\]


\end{document}
