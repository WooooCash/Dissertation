\documentclass[a4paper, 12pt]{report} \linespread{1.3}

\usepackage[T1]{fontenc} \usepackage[utf8]{inputenc} \usepackage{indentfirst}

\usepackage[english]{babel}

\usepackage[margin=3cm]{geometry}

\title{Skeletal Animation Using Inverse Kinematics in the Unity Engine}
\author{Łukasz Białczak} \date{}
	
\begin{document}

\maketitle

\tableofcontents

\chapter{Intro}

\textbf{Motive}

In standard skeletal animation the skeleton is represented by a tree-like
structure of transforms. Animation sequences are usually performed by updating
the position and rotation attributes of these transforms starting from the root
and propagating to the leaves. When applying inverse kinematics to skeletal
animation, as the name suggests, the transforms are updated starting from a leaf
and then back up the chain up to a selected node. This allows for a procedural
approach to creating animation sequences which better reflect a realistic
interaction between an animated object and its surroundings without the need to
manually bake the sequence. Interactions such as a character pressing a sequence
of buttons or pulling a lever are very well suited for the application of inverse
kinematics. Adjusting a characters limbs to uneven terrain is another popular
use for the technique. However, finding the proper parameter values for the
skeletal transforms is not always a trivial task and may require advanced
optimization methods. The purpose of this dissertation is to acquire a more
in-depth understanding of the basic algorithms used in inverse kinematics, as
well as discovering the built-in functionalities that the Unity engine offers
which are geared towards such implementations.


\textbf{Goal}

\textbf{Scope}

\chapter{Problem Analysis}
\begin{itemize}
    \item Describe the problem
    \item Don't mention the solution
    \item Assume the reader has no knowledge of the subject. Describe the
        problem domain and only then proceed to describe the comp sci version of
        the problem
\end{itemize}

\chapter{Existing Solutions}
\section{Overview}
\section{Comparisons}
\begin{itemize}
    \item Consider the pros and cons 
\end{itemize}
\section{Optimizations}

\chapter{Planned Solution}
\section{FABRIK algorithm overview}
\section{Advantages and Disadvantages}
\section{In-depth algorithm explanation}

\chapter{Project Design}
\begin{itemize}
    \item Apply knowledge from IO
\end{itemize}
\section{Project Structure}
\section{Implementation Strategy}
\section{Implementation Details}
\subsection{Scene Setup}
\subsection{Scripts}
\section{Additional Algorithm Modifications}
\section{Tests}

\chapter{Technologies Used}
\begin{itemize}
    \item Description of used technologies, should be critical and indicate as
        to why the given technology was chosen
    \item Follow up with a more in depth explanation on the technologies and why
        they were chosen
\end{itemize}

\section{Unity Engine}
\subsection{Scripting}
\subsection{Scene Hierarchy}
\subsection{Animation Rigging Package}
\begin{itemize}
    \item Implementation solutions
    \item Integration
    \item Tests
    \item User instructions
\end{itemize}

\chapter{Conclusion}
\begin{itemize}
    \item Conclusions - what was done
    \item Tie it back to the goals described at the beginning. "The goal was..."
        "... and these were the things that got done..."
    \item What didn't work out and potential improvements to be done in the
        future?
    \item Conclusion should be 1-2 pages
\end{itemize}

\section{Sources}


\end{document}
