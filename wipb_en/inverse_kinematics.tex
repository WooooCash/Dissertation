\chapter{Inverse Kinematics in the Unity Engine} 
The demo application written for the purpose of this dissertation includes two
separate use cases of skeletal animation using inverse kinematics in the Unity
engine. The first example is that of a four legged spider which uses the
technique as a means to more naturally adjust its limbs to the terrain it moves
around upon. The second example is the application of inverse kinematics to an
animation sequence of a human character pressing multiple buttons in succession.
The use of inverse kinematics allows the character to adjust its animation to hit
all the buttons without the need for a baked animation targeted towards each
button, as well as dynamically adjust the order of the buttons to be hit.

Although there are two separate use cases demonstrated in this application, both
use the same implementation of the FABRIK algorithm. 


\section{FABRIK implementation}
This implementation of the FABRIK algorithm is based on the paper written by the
author of the Unity engine FABRIK implementation. (CITE THE PAPER LMAO). For the
purposes of this application, the basic algorithm is implemented without many
additional constraints and options available. The one constraint added on to
this implementation is that of pole targets which will be further explained when
discussing the code behind them.

The script which implements the algorithm in this project takes in a few
parameters required to set up the mechanism. First and foremost, the root and
leaf nodes must be provided in order to define the kinematic chain which is to
be manipulated. The next object which the script must have knowledge of is the
target transform which the end effector will attempt to move to. All the
aforementioned parameters are required for the script to function. Additionally,
a pole target object may be optionally passed in if the use case requires it to
function in a manner that is desired.

\textit{show image of script parameters}

The first case which the algorithm must cover is if the distance from the root
to the target object is greater than the sum of distances between each adjacent
bone in the defined kinematic chain. In this case, the target is out of reach. Given
that the bones in a skeleton are expected to keep a fixed length, the end
effector will not be able to reach the target, and instead the kinematic chain
straightens and extends in the direction of the target. There is no need to
continue with the iterative portion of the algorithm.

In the implementation below, square magnitudes are used to
avoid the calculation of square roots, thus slightly optimizing the initial
check.

\begin{lstlisting}[basicstyle=\footnotesize, numbers=none,frame=single, caption={Target out of
reach},captionpos=b, label=stretch, language={[Sharp]c}]
void LateUpdate() 
{
    float distRootToTargetSqr = (
        target.position - chainRoot.position
    ).sqrMagnitude;
    if (distRootToTargetSqr > totalDistance * totalDistance)
    {
        StretchToTarget();
    }
    ...
}

void StretchToTarget()
{
    for (var i = 0; i < chainLen; i++)
    {
        float jointDistToTarget = (
            target.position - tempPositions[i]
        ).magnitude;
        var lambda = jointDistances[i] / jointDistToTarget;

        tempPositions[i + 1] =
            (1 - lambda) * tempPositions[i] + lambda * target.position;
    }
}
\end{lstlisting}

The scenario where the target is within the reach of the kinematic chain
utilizes iterative forward and backward component after which the algorithm is
named. Before each iteration, the positions of the joint transforms are copied.
All operations and calculations are performed on these copied transforms and at
the end of the full pass, the new positions are then copied back to the
kinematic chain. \textbf{CHECK IF THE ALGORITHM WORKS WITHOUT THIS. MIGHT SAVE
SOME MEMORY LMAO}. It is also important to note that the modification of the
transforms is done in Unity's \textit{LateUpdate} function. When using a mix of
inverse kinematics and baked animation, the object to which the IK script is
attached will have Unity's built-in \textit{Animator} component. If the IK
script updates joint transforms in the \textit{Update} method, then they may be
overwritten by the \textit{Animator} component. 

The forward pass of the algorithm iterates through the chain starting from the
end effector and ending at the root. At the start, the end effector's position
is set to be equal to the position of the target. A straight line is then drawn
from the end effector to the following node. This neighbours position is then
interpolated along the line so that the original distance between the two nodes
is kept. The same operation is performed for each pair of neighbouring nodes
throughout the pass. 

\textit{show code}

When the forward pass is complete, the root node is displaced from its original
position. This is undesired, as the root's node position should not be affected
by the algorithm. To remedy this, the next step is to repeat the forwards pass,
but this time in reverse. The root node's position is set equal to what it was
at the beginning of the frame. The next node is then interpolated between its
current position and the root to keep the initial bone length. As with the
forward pass, this is repeated for each subsequent pair of nodes. 

\textit{show code}

These two steps are repeated together until the end effector is within
a threshold distance of the target. The FABRIK algorithm is a heuristic
algorithm, and as such is does not lead to an exact result. Instead, it aims to
approximate the correct solution and solves the problem in a less complex and
more optimized way. Again, the square distances are used to avoid the
calculation of square roots. 

\textit{show code}

\textbf{POLE TARGETS. FIND A SOURCE WHICH EXPLAINS THE METHOD USED IN MY
PROGRAM. IF NOT THEN REFER TO THE YOUTUBE VIDEO LOLOL}


\section{Spider Movement}
The first use case for inverse kinematics in the demo application is that of
a four legged spider. The algorithm is used to adjust the creature's limbs to
uneven terrain, leading to a much more natural and realistic movement.
\subsection{Project Setup}
Each one of the spiders legs is treated as a separate kinematic chain. The
spider prefab consists of a container which holds the spider object itself, set
of empty objects to which the four IK scripts are attached for the purpose of
easy yet separated access. The prefab also contains sets of ray casts and
targets. The ray casts serve to scan the surface of the terrain under the
spider, and mark the targets to which each leg should move.

Ray casts are dispatched from above the spiders legs, and aim in the creatures
local negative Y axis. This ensures that no matter what orientation the spider
finds itself in, the rays are always pointing at the surface which it is
standing on. Masks are applied to the rays, making sure that only terrain
objects are taken into account, while the creature's body itself is not. The ray
cast hit point positions are then applied to each leg's respective target
object. These targets serve as markers for the limbs end effectors. 

\subsection{Scripts}
With the project set up in this manner, scripts must now be added on to make the
scene functional. A script is required for the main movement of the spider, the
ray cast logic, and the mechanism which controls the ik targets.

\subsubsection{Ray casts}
The ray cast objects contain a script component which dispatches the rays and
sets the appropriate target positions. First, a mask must be established, which
will then be passed into the ray cast operation. This is required so that only
terrain is counted as a valid hit. The lack of such a mask may result in
unexpected behavior, such as targets being set on the spider's body itself. The
ray cast object is then created, shooting in the local negative Y axis
direction. This ensures that no matter the orientation of the creature, the rays
are sent towards the surface that the spider is standing on. The targets
controlling the spider's legs will not be updating their positions to the
ray cast hit points each frame, though their scripts must have knowledge of the
hit positions at any given time. Given this, a separate set of objects are set
to track the ray cast hit points each frame.

\subsubsection{Target Logic}
In order for the spider's movement to seem realistic, the targets controlling
each leg must adhere to a set of rules pertaining to their movement. As
mentioned in the ray cast section, the IK targets cannot simply be set to track
the ray cast hit points. The following is an outline of the rules specified for
the IK targets, which define if it should start moving towards its ray cast hit
target:
\begin{itemize}
    \item A target must be grounded to be eligible for a movement sequence.

    \item A target will only begin moving towards the ray cast hit point if the
        distance between them is above a specified threshold.

    \item A target is only allowed to begin moving towards the ray cast hit
        point if both legs on the opposite diagonal are grounded.
        \textbf{CREATE A DIAGRAM AND REFERENCE IT HERE}
\end{itemize}

When a target satisfies all of these conditions, it makes note of the ray cast
hit's current position, which it will use for its upcoming movement sequence.
Once it begins moving, the \textit{Grounded} boolean is set to be false, so that
the other targets know whether they can begin moving or not.

The movement sequence itself is done using Unity's \textit{Vector3.MoveTowards}
method, which takes in a current position vector, a vector to move to, and the
maximum distance to move per frame, which can be used to control the movement
speed. This method allows the target to interpolate its position every frame.
The values fed into this method are simply the target's current position, the
ray cast hit target's position, which was recorded right before the beginning of
the movement sequence, and an arbitrary speed value, which is dependent on
\textit{Time.deltaTime} to avoid variations when the frame rate changes. The
only caveat is that in the first half of the movement, the destination vector's
height component is increased to achieve an arc-like movement. This produces the
effect of lifting the spider's leg. \textbf{FIX THIS IN CODE. USING WORLD Y-AXIS
CURRENTLY. CHECK THIS, MAYBE SINCE THE WHOLE CONTAINER IS ROTATED IT'S ACTUALLY
DONE RIGHT}

\subsubsection{General Movement}
The main movement script for the IK spider implementation brings the whole
system together. It has three main objectives:

\begin{itemize}

    \item Calculate the rotation of the spider so that it's local up and forward
        vectors can be set accordingly.

    \item React to input by moving the main body of the spider along with the
        ray casts.

    \item Regulate the height of the spiders body above the ground.

\end{itemize}

The first objective - calculating the rotation of the spider - is what allows it
to scale walls and walk upside down on the ceiling. The rotation is determined
based on the limb positions at the beginning of each \textit{Update} call.
First, two vectors are constructed from the end effectors of both sets of
diagonally opposed legs. The local up vector of the spider is then calculated by
taking the cross product of these two vectors. This determines the orientation
of the spider's main body, and it also affects the direction that the four legs'
rays are cast. The local forward vector is then obtained from the cross product
between the newfound up vector and the spider's right vector. This new vector is
what will be used to determine the direction of movement when the spider
receives input from the user.

The second objective - reacting according to input - is quite simple once the
local directional axes are determined. When the script detects a non-zero value
on either the \textit{Horizontal} or \textit{Vertical} axis, it reacts
accordingly by moving the spider along with all the ray cast objects. The
\textit{Vertical} axis corresponds to the forward and backward movement of the
spider. The script reacts by simply moving the spider's main body along the
local forward axis mentioned previously. Additionally, the ray casting objects
are offset either forwards or backward depending on the direction of the
movement. This is done because the ray casts must be slightly ahead of the
default leg positions so that the legs end up moving in a natural manner.
The \textit{Horizontal} axis is responsible for rotating the spider about it's
local up axis which allows the spider to turn around. 

Finally, the spider's height off of the surface must be regulated each frame.
The lack of a gravitational force acting on the body to keep it flush with the
ground, and the unconstrained rotational capability of the creature, means that
the distance between the spider and the surface it is walking on must be
procedurally kept in check. This is done with yet another ray cast which
originates from the center of the spider's body and points in the negative up
axis direction. If the distance to the hit point exceeds an acceptable range,
the body is moved towards said range, again using the
\textit{Vector3.MoveTowards} method to linearly interpolate the spiders position
and avoid excessive jerkiness which occurs with a frequent variation in height. 



\section{Human Animation Sequence}
\subsection{Project Setup}
\subsection{Scripts}
