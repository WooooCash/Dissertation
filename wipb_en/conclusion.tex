\chapter{Conclusion} 
The aim of this paper was to gain a better understanding of the basic algorithms
used in inverse kinematics, to discover the built-in functionalities that the Unity
engine offers for such implementation, and compare the IK procedural animations
to a baked animation approach through visual and performance based
metrics. The aim was achieved. 

In the second chapter, the theory behind existing IK
algorithms was analyzed. As a result of this analysis, the heuristic FABRIK
algorithm was chosen for its speed and simple implementation which distinguishes
itself from other approaches by avoiding problems such as instabilities and
complex calculations.

In the third chapter, an overview was made of technologies used in the
implementation of the demo application. Among others, the features provided by
the Unity game engine which enable and facilitate the use of IK were outlined.

The third and fourth chapters served to display the implementation details of
both the IK solutions and their baked animation counterparts, as well as compare
them in the visual and performance categories. In terms of visual realism and
natural motion, the IK animations proved to be superior in terms of precise
interaction with the surrounding environment, and the adaptability of the
characters to various situations. The human animation sequence demonstrated that
the procedural animations using inverse kinematics result in a higher CPU usage
compared to baked animations. However, the spider movement animation countered
this by showing that in certain cases, the procedural approach to animation and
movement can eliminate the need for various calculations such as physics checks,
and can overall result in a lower net CPU usage compared to a character which
has baked animations.
