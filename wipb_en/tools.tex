\chapter{Tools} 
Multiple tools were used in the process of creating the demo application for
this paper including the Unity game engine, Blender as a modeling and animation
software, and MakeHuman as a model creationi tool. This chapter discusses the
built-in functionalities which make the mentioned tools an effective choice.

\section{MakeHuman}
MakeHuman is an open source tool for making 3D characters. It provides
a convenient way of acquiring a human model which is customizeable and can be
exported in various formats in order to be used in other softwares. The key
factors which make this tool suitable for use in the demo application is the
options it provides regarding the complexity of the topology of the model's
mesh, and the choice of skeleton rig alongside the fact that the exported model
is already rigged and ready to be used in an animation software. One of the rig
options is specifically designed to be then used in a game engine setting (See
Figure ).

\section{Unity}
The Unity game engine is the one tool which was non-negotiable as the paper
is meant to specifically focus on the usage of inverse kinematics in said game
engine. Nevertheless, the engine is a good seleciton for this use case due to
its advanced 3D support, the built-in packages and functionalities which are
geared towards the subject of this paper, and the overall popularity of the
engine and large community built around it which results in a substantial amount
of documentation and support. 

\subsection{Importing Animations}

\subsection{Animator Controller}

\subsection{Animation Rigging Package}
