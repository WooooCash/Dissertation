\chapter{Related Work}
Inverse kinematics is applied in various fields such as robotics and computer
graphics. Over time, many approaches have surfaced in order to solve the IK
problem. There are multiple families of solutions \cite{Aristidou2011} which
suit different use cases. Among others, trigonometric solutions can be used to
solve certain IK problems, however, these solutions are often limited to solving
two bone IK scenarios. 

\section{IK Algorithms}
