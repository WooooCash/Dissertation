\chapter{Introduction}
\section{Motivation} 
Animation is the technique of displaying different positions of a character or
object in rapid succession to create the illusion of movement. It is used in
various forms of entertainment, such as movies and video games. In the latter,
unlike in the former, the animation sequences are performed in real time and
therefore impose additional constraints. Without the freedom to process a single
frame for minutes or hours during the rendering of the scene, the animator must
compromise on the quality and realism of the sequence in order to optimize for
gameplay. Furthermore, the interactive nature of video games makes it impossible
for the artist to create predefined animation sequences for every possible
situation that may occur in the game. As a result, predefined animation
sequences are often generic and do not allow the character or object to interact naturally with their surroundings.

Game developers have come up with many methods to improve the realism of
animation in games such as playing cutscenes for critical interactions between
a character and the world. However, this paper will focus on the use of
procedural animation and, more specifically, the application of inverse
kinematics to skeletal animations in video games.
\section{Problem Formulation}

Skeletal animation is a popular technique for animating character models in
computer graphics. Animation sequences are preformed by manipulating a tree-like
structure of interconnected bones, represented by transforms, to create the
desired motion of the character. In standard skeletal animation, these
manipulations are done starting from the root node to a leaf node. When applying
inverse kinematics, they are instead done from the leaf node to a selected
ancestral node. The technique calculates the joint angles required to achieve
the desired position of the end effector. This dissertation's aim is to gain
a better understanding of the basic algorithms used in inverse kinematics, as
well as discover the built-in functionalities that the Unity engine offers for
such implementation.
