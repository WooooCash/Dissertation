\chapter{Introduction}
\section{Motivation}
\section{Problem Formulation}
In standard skeletal animation the skeleton is represented by a tree-like
structure of transforms. Animation sequences are usually performed by updating
the position and rotation attributes of these transforms starting from the root
and propagating to the leaves. When applying inverse kinematics to skeletal
animation, as the name suggests, the transforms are updated starting from a leaf
and then back up the chain up to a selected node. This allows for a procedural
approach to creating animation sequences which better reflect a realistic
interaction between an animated object and its surroundings without the need to
manually bake the sequence. Interactions such as a character pressing a sequence
of buttons or pulling a lever are very well suited for the application of inverse
kinematics. Adjusting a characters limbs to uneven terrain is another popular
use for the technique. However, finding the proper parameter values for the
skeletal transforms is not always a trivial task and may require advanced
optimization methods. The purpose of this dissertation is to acquire a more
in-depth understanding of the basic algorithms used in inverse kinematics, as
well as discovering the built-in functionalities that the Unity engine offers
which are geared towards such implementations.
