\chapter{Introduction}
\section{Motivation} 
Animation is the technique of displaying different positions of a character or
object in rapid succession to create the illusion of movement. It is used in
various forms of entertainment, such as movies and video games. In the latter,
unlike in the former, the animation sequences are performed in real time and
therefore impose additional constraints. Without the freedom to process a single
frame for minutes or hours during the rendering of the scene, the animator must
compromise on the quality and realism of the sequence in order to optimize for
gameplay. One such optimization is the use of skeletal animation in which
animation sequences are performed by manipulating a tree-like structure of
interconnected bones, represented by transforms, to create the desired motion of
the character. Furthermore, the interactive nature of video games makes it
impossible for the artist to create predefined animation sequences for every
possible situation that may occur in the game. For example, an animation of
a hand pressing a button is only valid if the character to which the hand
belongs is placed exactly in a predefined position. If the button changes its
size or position, the baked animation must be altered, or it simply loses its
realism. As a result, predefined animation sequences are often generic and
do not allow the character or object to interact naturally with their
surroundings. Game developers have come up with many methods to improve the
realism of animation in games such as playing cutscenes for critical
interactions between a character and the world. However, this paper will focus
on the use of procedural animation and, more specifically, the application of
inverse kinematics to skeletal animations in video games.

Inverse kinematics (IK) is a technique used in fields such as robotics and computer
graphics to determine the joint angles of a kinematic chain that will result in
a particular part of the chain, usually an end effector, reaching a specified
position in 3D space. In computer graphics specifically, the technique is often
used to animate the movement of characters and objects such that they interact
with their surroundings in a more realistic manner. Inverse kinematics is even
used in modeling and animation software such as Blender \cite{blender_ik2} to
speed up the process of limb manipulation when creating a static animation.

There are multiple approaches and algorithms that exist within the inverse
kinematics domain, such as analytical methods, gradient descent, and
optimization techniques. The choice of approach varies depending on the
complexity of the use case, the desired realism of the animation, and system
limitations.

\section{Problem Formulation}

The aim of this paper is to gain a better understanding of the basic algorithms
used in inverse kinematics and to discover the built-in functionalities that the
Unity engine offers for such implementation. The project implementation will
apply these concepts to create pairs of animations which consist of baked and
inverse kinematics variants. The use cases will expand the problem by
introducing additional constraints which will be required to keep the
consistency and realism of the animations. The variations will then be compared
through the lens of realism and performance.

The author will begin by discussing the theory of the different approaches and
algorithms used to solve the inverse kinematics problem, and the resulting
choice of the algorithm to be used in the project implementation. The following
sections will explain in depth the implementation of two use cases which
demonstrate the purpose of inverse kinematics as a skeletal animation technique.
Experiments will then be conducted to compare the inverse kinematics animations
with their baked counterparts based on realism and performance. Finally,
a summary and conclusion of important points will be presented.
