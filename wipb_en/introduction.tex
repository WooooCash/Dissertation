\chapter{Introduction}
\section{Motivation} 
Animation is the technique of displaying different positions of a character or
object in rapid succession to create the illusion of movement. It is used in
various forms of entertainment, such as movies and video games. In the latter,
unlike in the former, the animation sequences are performed in real time and
therefore impose additional constraints. Without the freedom to process a single
frame for minutes or hours during the rendering of the scene, the animator must
compromise on the quality and realism of the sequence in order to optimize for
gameplay. Furthermore, the interactive nature of video games makes it impossible
for the artist to create predefined animation sequences for every possible
situation that may occur in the game. As a result, predefined animation
sequences are often generic and do not allow the character or object to interact
naturally with their surroundings.

Game developers have come up with many methods to improve the realism of
animation in games such as playing cutscenes for critical interactions between
a character and the world. However, this paper will focus on the use of
procedural animation and, more specifically, the application of inverse
kinematics to skeletal animations in video games.
\section{Problem Formulation}
In standard skeletal animation, the skeleton is represented by a tree-like
structure of transforms. Animation sequences are typically performed by updating
the position and rotation attributes of these transforms, starting from the root
and propagating to the leaves. When applying inverse kinematics to skeletal
animation, the transforms are updated starting from a leaf and then back up the
chain up to a selected node, allowing for a procedural approach to creating
animation sequences that better reflect a realistic interaction between an
animated object and its surroundings without the need to manually bake the
sequence. Examples of interactions that are well-suited for the application of
inverse kinematics include a character pressing a sequence of buttons or pulling
a lever, or adjusting a character's limbs to uneven terrain. However, finding
the proper parameter values for the skeletal transforms is not always a trivial
task and may require advanced optimization methods. This dissertation aims to
acquire a more in-depth understanding of the basic algorithms used in inverse
kinematics, as well as discover the built-in functionalities that the Unity
engine offers for such implementations.
