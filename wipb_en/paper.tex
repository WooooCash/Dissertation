\documentclass[oneside,12pt]{wipb}
\usetikzlibrary{mindmap,trees}%dla diagramu Computer science mindmap
\usepackage{amsmath}
\usepackage{amssymb}
\usepackage{bm}
\usepackage[english]{babel, fancyref, cleveref}
\usepackage{url}
\usepackage{subcaption}
\usepackage[font=footnotesize]{caption}
\usepackage{array}
\usepackage{courier}

\usepackage{chngcntr}
\counterwithout{equation}{subsection}
\counterwithout{equation}{section}


\newcommand{\pvec}[1]{\vec{#1}\mkern2mu\vphantom{#1}}

\katedra{Biometrii}
\typpracy{ %magisterska
           inżynierska
         } 
\temat{Skeletal Animation Using Inverse Kinematics in the Unity Engine}
\autor{Łukasz Jan Białczak}
\promotor{dr inż. Adam Borowicz}

\hypersetup{ %wpisy w pdf info
pdfauthor={Łukasz Jan Białczak},
pdftitle={Skeletal Animation Using Inverse Kinematics in the Unity Engine},
pdfsubject={},
pdfkeywords={Praca dyplomowa},
pdfpagemode=UseNone,
linkcolor=black,
citecolor=black
} 
\begin{document}
\maketitle
\thispagestyle{empty}
\chapter*{}
% \noindent\textbf{Animacja szkieletowa z wykorzystaniem kinematyki odwrotnej w silniku Unity}

% W klasycznej animacji szkieletowej kości reprezentowane są za pomocą pewnej
% struktury drzewiastej obiektów. Wykonywanie sekwencji animacji polega zwykle na
% aktualizacji parametrów tych obiektów (położenia i orientacji kości) w porządku
% od korzenia do liści. W przypadku kinematyki odwrotnej, jak sama nazwa wskazuje,
% porządek aktualizacji jest odwrotny tj. od danego liścia do określonego węzła
% nadrzędnego. Metody te pozwalają na proceduralne tworzenie sekwencji animacji
% umożliwiających wierniejsze odzwierciedlenie interakcji animowanego obiektu
% z otoczeniem bez konieczności ręcznego "wypalania" sekwencji animacji. W ramach
% tej pracy, autor zapoznał się z teorią i literaturą na temat animacji
% szkieletowej i kinematyki odwrotnej, a następnie zaimplementował aplikację
% demonstracyjną w silniku Unity. Aplikacja ma na celu porównanie animacji
% "wypalanych" z animacjami tworzonymi z wykorzystaniem kinematyki odwrotnej.
% Zostało to zrealizowane na przykładzie czworonoga, i sekwencji animacji
% człowieka. Wyżej wymienione techniki są porównane pod kątem efektu wizualnego
% oraz wydajnościowego. 

\noindent Subject of diploma thesis

\noindent\textbf{Skeletal Animation Using Inverse Kinematics in the Unity Engine}

\begin{center}
    SUMMARY
\end{center}

In standard skeletal animation, the skeleton is represented by a tree-like
structure of transforms. Animation sequences are typically performed by updating
the position and rotation attributes of these transforms in order, from the root
to the leaves. When applying inverse kinematics to skeletal animation, the
transforms are updated in reverse order, starting from a leaf and then working
back up the chain to a selected node. This allows for a procedural approach to
creating animation sequences which better reflect interactions between an
animated object and environment, without the need to manually bake the animation
sequence. For the purpose of this work, the author analyzed theory and
literature concerning skeletal animation and inverse kinematics, and then
implemented an application in the Unity engine. The aim of the application is to
compare baked animations with procedural animations which make use of inverse
kinematics. This was demonstrated using examples of a spider, and a human
character's animation sequence. The two animation methods were compared in their
visual quality and performance.
\thispagestyle{empty}

\tableofcontents
\addtocontents{toc}{\protect\thispagestyle{empty}}
\thispagestyle{empty}
\setcounter{page}{0}
\pagestyle{plain}
\chapter{Introduction}
\section{Motivation} 
Animation is the technique of displaying different positions of a character or
object in rapid succession to create the illusion of movement. It is used in
various forms of entertainment, such as movies and video games. In the latter,
unlike in the former, the animation sequences are performed in real time and
therefore impose additional constraints. Without the freedom to process a single
frame for minutes or hours during the rendering of the scene, the animator must
compromise on the quality and realism of the sequence in order to optimize for
gameplay. Furthermore, the interactive nature of video games makes it impossible
for the artist to create predefined animation sequences for every possible
situation that may occur in the game. As a result, predefined animation
sequences are often generic and do not allow the character or object to interact
naturally with their surroundings.

Game developers have come up with many methods to improve the realism of
animation in games such as playing cutscenes for critical interactions between
a character and the world. However, this paper will focus on the use of
procedural animation and, more specifically, the application of inverse
kinematics to skeletal animations in video games.
\section{Problem Formulation}
In standard skeletal animation, the skeleton is represented by a tree-like
structure of transforms. Animation sequences are typically performed by updating
the position and rotation attributes of these transforms, starting from the root
and propagating to the leaves. When applying inverse kinematics to skeletal
animation, the transforms are updated starting from a leaf and then back up the
chain up to a selected node, allowing for a procedural approach to creating
animation sequences that better reflect a realistic interaction between an
animated object and its surroundings without the need to manually bake the
sequence. Examples of interactions that are well-suited for the application of
inverse kinematics include a character pressing a sequence of buttons or pulling
a lever, or adjusting a character's limbs to uneven terrain. However, finding
the proper parameter values for the skeletal transforms is not always a trivial
task and may require advanced optimization methods. This dissertation aims to
acquire a more in-depth understanding of the basic algorithms used in inverse
kinematics, as well as discover the built-in functionalities that the Unity
engine offers for such implementations.

\chapter{Related Work and Theory}
Inverse kinematics is applied in various fields, mainly robotics and computer
graphics. However, it is applied to a few unique problems, such as the
prediction of protein structure \cite{ccd_protein}. It has also found its uses in
rehabilitation medicine due to its miomechanical modelling ability. Over time,
many approaches have surfaced in order to solve the IK problem. There are
multiple families of solutions \cite{Aristidou2011} which suit different use
cases. Among others, trigonometric solutions can be used to solve certain IK
problems analytically, however, these solutions are often limited to solving two
bone IK scenarios. This paper will go over the iterative and heuristic
approaches which provide less complex solutions with the expansion of kinematic
chain lengths and degrees of freedom.

\section{Kinematics}
Kinematics is a branch of physics and a subdivision of classical mechanics
concerned with the geometrically possible motion of a body or system of bodies
without consideration of the forces involved \cite{kinematics_britannica}.
A kinematic chain is a tree like hierarchical structure of joint transforms
which are connected. Because the relationship between joints is hierarchical,
a transformation which is applied to a given joint affects all of its descendant
nodes. When manipulating a kinematic chain, transformations in the form of
rotations and translations are applied to joint transforms in order to achieve
a desired position of one or multiple transforms called the end effectors. The
problem of kinematics can be subdivided into two approaches which each present
their own problem and solution.

\begin{itemize}
    \item The problem of \textit{Forward Kinematics} which has to do with the
        identification of the final position of an end effector as a result of
        a set of transformations being applied to the kinematic chain to which
        the end effector belongs.
    \item The problem of \textit{Inverse Kinematics} which pertains to the
        search for a configuration of joint transformations for kinematic
        chain which allow the end effector to reach a predefined target
        position.
\end{itemize}

The forward kinematics (FK) problem can be said to have a guaranteed solution as
long as the set of joint transformations used as an input are known. To solve
the FK problem, the transformations are applied to the kinematic chain, and the
transform of the end effector is taken as an output. On the contrary, when
dealing with IK, the given problem may have no solutions, one solution, or many
solutions. 

\subsubsection{No Solutions}
An IK problem with no solutions can occur for several reasons. Most notably, and
IK solution is impossible when the target is unreachable through the limitations
of chain length. When the distance between the chain root and the target is
larger than the sum of distances between each adjacent chain link (Figure
\ref{fig:unreachable_dist1}), it is impossible to achieve a solution, as the
root of the chain is static and cannot be moved in order to allow the end
effector to reach its target. Similarly, and unreachable case exists if the
first bone segment is longer than the sum of the remaining bone segments (Figure
ref).

\begin{figure}[!h]
    \centering
    \captionsetup{justification=centering}
    \includegraphics[width=0.5\textwidth]{grafika/unreachable_dist_1.png}
    \caption{An IK problem with no solution where the target is unreachable
    because the sum of segment distances is less than the distance between the
    root and the target \(\sum_{i=1}^{n}d_i < dt\) }
    \label{fig:unreachable_dist1}
\end{figure}

\begin{figure}
    \centering
    \captionsetup{justification=centering}
    \includegraphics[width=0.4\textwidth]{grafika/unreachable_dist_2.png}
    \caption{An IK problem with no solution where the target is unreachable
    because the length of the first segment is greater than the summed length of
    the rest of the kinematic chain \(d_1 - \sum_{i=2}^{n}d_i > dt\). This creates
    a radius around the root where if its distance to the target is smaller, it
    indicates an unreachable target.  } \label{fig:unreachable_dist2}
\end{figure}

Another reason for the lack of solutions to an IK problem can be the
over-constraining of the problem. Constraints in an IK problem are used to
dictate the range of motion that each joint. Constraints might limit the joint's degrees of
freedom by reducing the dimensions in which it can perform its rotational and
translational transformations. Such constraints are often modelled after joints
which are present in the domain of biomechanics, such as hinge joints, ball
joints, pivot joints. A constraint can also limit the extent of a joint's
ability to bend which is defined by the allowed angle of the joint's rotation in
relation to the rotation of its parent node. If too many of such constraints are
added to a kinematic chain, it may have blind spots for which it is unable to
find a configuration of transformations which can bend the chain to reach the
target.

\begin{figure}
    \centering
    \captionsetup{justification=centering}
    \includegraphics[width=0.4\textwidth]{grafika/unreachable_angles.png}
    \caption{An IK problem with no solution where the rotational constraints
    placed on the joints prevent the end effector from being able to bend enough
    to reach the target}
    \label{fig:params}
\end{figure}

\subsubsection{One or many solutions}
When an IK problem is not limited by the cases mentioned above, it can have one,
or many solutions depending on the case and constraints. More often than not the
problem will have multiple available solutions, and if the kinematic chain is
unconstrained, the solution set for and IK problem grows very large.

While the introduction of constraints can increase the cases for which an IK
problem has only one solution, the simplest case to consider for an
unconstrained kinematic chain is one where the chain must stretch to its full
extent in order to reach the target. The sum of the chain's segment lengths
\(d_i\) is equal to the distance from the root to the target \(dt\).
\[
    \sum_{i=1}^{n}d_i = dt
    \]

\section{IK Algorithms}

\chapter{Tools} 
Multiple tools were used in the process of creating the demo application for
this paper including the Unity game engine, Blender as a modeling and animation
software, and MakeHuman as a model creation tool. This chapter discusses the
built-in functionalities which make the mentioned tools an effective choice.

\section{MakeHuman}
MakeHuman is an open source tool for making 3D characters. It provides
a convenient way of acquiring a human model which is customizable and can be
exported in various formats in order to be used in other software programs. The
key factors which make this tool suitable for use in the demo application is the
options it provides regarding the complexity of the topology of the model's
mesh, and the choice of skeleton rig alongside the fact that the exported model
is already rigged and ready to be used in an animation software. One of the rig
options is specifically designed to be then used in a game engine setting
(Figure \ref{fig:mh_rig}).

\begin{figure}
    \centering
    \includegraphics[width=\textwidth]{grafika/make_human_rig.png}
    \caption{MakeHuman rig selection}
    \label{fig:mh_rig}
\end{figure}

\section{Blender}
The tool of choice for modeling and animation used for the demo application is
Blender. It is a free and open source tool offering a suite of functionalities
including the creation of 3D models, rigging, and animation. 

% The program enables
% users to import models in various formats, which allows the user to use
% externally generated models, such as the ones created in MakeHuman, and animate
% them. 

Blender offers the functionality of importing existing models in various formats
including the \textit{collada} format \cite{collada} which is the default export
option in MakeHuman. The models, which are already rigged, can then be animated
using the Blender animation pose editor. 

Models can also be created from scratch. Blender offers a 3D modelling tool to
create a desired mesh. A custom rig can also be constructed and attached to the
created model. Weights can be painted on the meshes vertices for each bone to
define how tightly they are bound, and how much the position of each vertex
depends on the given bone position. 

Lastly, an animation sequence can be created for an existing mesh and rig using
the character animation pose editor. The user can define poses for different
points in time by creating key frames on a timeline, and blender interpolates the
bone positions in between the key frames. This is used to create baked animations
for characters and objects, as well as defining animations that are later
blended with IK constraints. An animated model can be then exported in the
\textit{fbx} format to be used in other software programs. Unity also supports
importing a model from a \textit{.blend} file which is the extension of
a blender project file. 


\section{Unity}
The Unity game engine is the one tool which was non-negotiable as the paper
is meant to specifically focus on the usage of inverse kinematics in said game
engine. Nevertheless, the engine is a good selection for this use case due to
its advanced 3D support, the built-in packages and functionalities which are
geared towards the subject of this paper, and the overall popularity of the
engine and large community built around it which results in a substantial amount
of documentation and support. 

\subsection{Importing Animations}

\subsection{Animator Controller}

\subsection{Animation Rigging Package}

\chapter{Inverse Kinematics in the Unity Engine} 
\section{FABRIK implementation}
\section{Spider Movement}
\subsection{Project Setup}
\subsection{Scripts}
\section{Human Animation Sequence}
\subsection{Project Setup}
\subsection{Scripts}

\chapter{Experiments}
The main purpose of this chapter of this paper is to compare animations
generated procedurally with the use of Ik with baked animations. The comparison
is broken up into two categories - visual and performance. 

\section{Baked animations}
 In order to conduct the experiments comprised of visual and performance
 comparisons, a second set of animations were created each of the examples which
 are shown in the demo application. Below is an explanation of the process of
 creating this second set of animations.

\subsubsection{Spider}
The baked animation version of the spider was animated in Blender. The set
consists of idle and walking animations which, in Blender, are placed on
a single timeline, one after the other (Figure \ref{fig:timeline}). Once the model is imported
into Unity, the animations can be broken up into their separate cases, as shown
earlier in the "tools" chapter of this paper (Figure \ref{fig:anim_chunk}). 

\begin{figure}[h!]
    \centering
    \includegraphics[width=0.7\textwidth]{grafika/blender_timeline.png}
    \caption{Two animations on a single timeline}
    \label{fig:timeline}
\end{figure}

Once the animations have been imported, a new animation controller is created
for the new spider. The animations are added, this time with the use of a blend
tree (Figure \ref{fig:s_blendtree}) to, as the name suggests, blend between the
animation smoothly CITE. A third animation state is added to the blend tree
which is the equivalent of the walking animation, but the animation speed is set
to a negative value. This plays the animation in reverse and is used when the
spider is walking backwards. 

\begin{figure}[h!]
    \centering
    \includegraphics[width=0.7\textwidth]{grafika/spider_blend.png}
    \caption{Blend tree containing animations for the spider}
    \label{fig:s_blendtree}
\end{figure}

A variable named \textit{spider\_movement} is created in order to control the
animations that are to be played in a given situation, and the manner in which
the transitions should be blended. The thresholds for each animation can be
defined in the blend tree's configuration in the animator (Figure ref).

\begin{figure}[h!]
    \centering
    \captionsetup{justification=centering}
    \includegraphics[width=0.7\textwidth]{grafika/spider_blendconf.png}
    \caption{Configuration of the spider's blend tree which is dependent on
    the \textit{spider\_movement} variable}
    \label{fig:s_blendconf}
\end{figure}

In the spiders movement script, this variable can then be set in reaction to
certain inputs so that the proper animation is activated each given situation.
To achieve the transition blending, the \textit{spider\_movement} variable
should not be set outright to the value which corresponds to the next animation
state. Instead, the \textit{\_animator.SetFloat} method is used, where the
\textit{\_animator} is a reference to the spider's Animator component. This
method allows the value of spider\_movement to be interpolated from the value of
the current animation state to another desired value.
\newline
\begin{lstlisting}[basicstyle=\footnotesize, numbers=none,frame=single,
caption={Transitioning to the spider's idle animation using the
\textit{SetFloat} method},captionpos=b, label=stretch, language={[Sharp]c}]
    if (verticalAxis == 0f && horizontalAxis == 0f)
        _animator.SetFloat("spider_movement", 0f, 0.05f, Time.deltaTime);
\end{lstlisting}

This version of the spider has its movement based on the spider from the game
Minecraft, which means that its rotation on the x and z axes is locked. The
spider can rotate about the y-axis when turning to face a different direction,
but it will not adjust to variations in the surface which it walks upon.
Additionally, when the spider encounters a vertical wall, it begins moving
vertically instead of horizontally until it scales the entire obstacle.

\subsubsection{Human}
The baked version of the human character, which performs an animation sequence
consisting of pressing buttons, is also animated in Blender. The animation plays
one full sequence of pressing a single button. Chunks of the same animation are
used in the IK version of this animation sequence which is described in the
previous chapter, however the baked animation utilizes the full animation while
the IK version uses only the beginning and the end. Nevertheless, the animation
is still broken up into three parts when imported into Unity: the raising of the
hand, the button pressing motion, and the lowering of the hand (Figure
\ref{fig:bp_clips}). This is done because when the character is pressing
multiple buttons in a row, the hand should not be lowered to its starting
position after every press.

\begin{figure}[h!]
    \centering
    \captionsetup{justification=centering}
    \includegraphics[width=0.7\textwidth]{grafika/bp_clips.png}
    \caption{The full button press animation is broken up into 3 separate clips}
    \label{fig:bp_clips}
\end{figure}

Another animation controller is created for this version of the human. Unlike
the spider example, the animation states do not have to be blended as they are
all clips which combine to create the full button press animation, and the
transitions are seamless as they are. Because of this, the animation states for
the three animation clips are not part of a blend tree, and instead are just
"floating" states with no defined transitions. The logic for which animation
should be played at what time is defined in the main script attached to the
character. 

The script is a simplified version of the one which controls the IK version of
this character's animation. It has no need for the public parameters present in
it's IK version because there is no IK rig, IK target, or button transforms
which it needs to control. Due to this, there is no list containing the sequence
of buttons to be pressed. It is instead replaced by an integer value which
dictates the number of button presses to execute in one animation sequence in
order to convey the idea that the character is pressing multiple buttons in
a sequence. 

As with the IK version of this script, the logic is based on a set of coroutines
which control the flow of the animation sequence. However, only the
\textit{HandAnimation} coroutine is used as a building block for the sequence
because the whole action is now constructed using baked animations. When the
script receives an input and the animation is not already playing, the sequence
begins by setting the \textit{idle} variable to true to prevent the sequence
from being repeated while it is still in progress. All required animation clips
are then set off one after the other, starting with the animation to lift the
hand. This is then followed by the animation clip which is responsible for
hitting the button, and it is repeated in a loop for a number of times defined
by the button press count parameter. Finally, the animation clip for lowering
the hand is executed, and the \textit{idle} boolean is set to false before
terminating the sequence.

\section{Visual Comparison}

\subsubsection{Spider}
\subsubsection{Human}

\section{Performance Comparison}

\subsubsection{Spider}
\subsubsection{Human}

\chapter{Conclusion} 
The aim of this paper was to gain a better understanding of the basic algorithms
used in inverse kinematics, to discover the built-in functionalities that the Unity
engine offers for such implementation, and compare the IK procedural animations
to a classic baked animation approach through visual and performance based
metrics. The aim was achieved. 

In the second chapter, the theory behind existing IK
algorithms was analyzed. As a result of this analysis, the heuristic FABRIK
algorithm was chosen for its speed and simple implementation which distinguishes
itself from other approaches by avoiding problems such as instabilities and
complex calculations.

In the third chapter, an overview was made of technologies used in the
implementation of the demo application. Among others, the features provided by
the Unity game engine which enable and facilitate the use of IK were outlined.

The third and fourth chapters served to display the implementation details of
both the IK solutions and their baked animation counterparts, as well as compare
them in the visual and performance categories. In terms of visual realism and
natural motion, the IK animations proved to be superior in terms of precise
interaction with the surrounding environment, and the adaptability of the
characters to various situations. The human animation sequence demonstrated that
the procedural animations using inverse kinematics result in a higher CPU usage
compared to baked animations. However, the spider movement animation countered
this by showing that in certain cases, the procedural approach to animation and
movement can eliminate the need for various calculations such as physics checks,
and can overall result in a lower net CPU usage compared to a character which
has baked animations.



% \nocite{*} %wszystkie wpisy w bibliografi
\bibliographystyle{IEEEtranS} %{latex8} posortowane wzgledem wystepowania
\urlstyle{same}
\bibliography{bibliografia}%

\listoffigures
\listoftables
\lstlistoflistings


\end{document}
